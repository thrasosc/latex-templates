\documentclass[11pt, oneside]{article}
\usepackage[margin=1in]{geometry}
\geometry{letterpaper}
\usepackage[parfill]{parskip}
\usepackage{graphicx}
\usepackage{float}
\usepackage[svgnames]{xcolor}
\usepackage{enumitem}
\usepackage[hidelinks]{hyperref}
\usepackage{amssymb}
\usepackage{amsmath}
\usepackage{lipsum} % For Lorem Ipsum text

% Font setup
\usepackage{mathpazo} % Palatino for text and math
\linespread{1.05}      % Palatino needs more leading
\usepackage[scaled]{helvet} % Helvetica for sans-serif
\usepackage{courier}   % Courier for typewriter font
\normalfont
\usepackage[T1]{fontenc}

% Set up hyperref to use the defined colours
\hypersetup{
    colorlinks=true,
    linkcolor=black,
    urlcolor=blue,
    citecolor=red
}

% Title Information
\title{Your Document Title}
\author{Your Name Here\\ Your Student ID Here}
\date{\today}  % Automatically sets the current date. You can replace this with a fixed date if needed.

\begin{document}
\maketitle

\tableofcontents % Generate table of contents
\newpage

\section{Section Title}
% Add your content here.
\lipsum[2-4]  % Placeholder Lorem Ipsum text for filler content.

\newpage

\section{Mathematics Example}
Here’s an example of some inline math: $a^2 + b^2 = c^2$.

And here’s an equation on its own:

\[
E = mc^2
\]

\newpage

\section{Conclusion}
This section can contain your concluding remarks or summary of the assignment.

\end{document}
